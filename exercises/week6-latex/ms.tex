\documentclass{article}
\usepackage[utf8]{inputenc}
\usepackage{graphicx}
\title{\LaTeX{} and Gnuplot: the Gamma function}
\author{Esben Hygum Larsen}
\date{\today}

\begin{document}
\maketitle
The gamma function is defined as:
\begin{equation}
    \Gamma (n)=(n-1)!\ .
\end{equation}
\noindent where n is a positive integer. It can also be defined for a complex number with a positive real part:

\begin{equation}
    \Gamma (z)=\int _{0}^{\infty }x^{z-1}e^{-x} dx, \qquad \Re (z)>0
\end{equation}
\noindent It is an extension of the factorial function, taking complex and real number arguments. There are no points at which the Gamma function is equal to zero, but it does diverge at $x = 0, \quad -1, \quad -2, \quad -3, \quad -4, \quad ...$ it which it is not anaytical.


\begin{figure}
    \centering
    \input{plot-gamma.tex}
    \caption{Gamma function}
\end{figure}

\end{document}
